\documentclass[numerate]{cheatsheet}
\usepackage{blindtext}
\usepackage{multicol}
\usepackage{tabularx}
\usepackage{empheq}
\usepackage{makecell}
\usepackage{xcolor}
\usepackage{scalerel}
\usepackage{amsmath}
\usepackage{graphicx}
\usepackage{wrapfig}
\usepackage{wasysym}
\usepackage{amssymb}
\usepackage{color,soul}
\usepackage{colortbl}
\usepackage{multirow}
\usepackage{bigstrut}
\usepackage{gensymb}
\usepackage{hhline}
\usepackage{mathtools}

\usepackage[center]{}



\doctitle{Zusammenfassung \\Mechanik II}
\author{Nina Schmidt - nischmidt$@$ethz.ch}


\begin{document}

\renewcommand{\multirowsetup}{\centering} %to align center inside the table
\setlength{\tabcolsep}{4pt}
%New command
\newcommand{\cbreak}{%
    \vfill\null\columnbreak%
}

%New command
\makeatletter
\newcommand{\pushright}[1]{\ifmeasuring@#1\else\omit\hfill$\displaystyle#1$\fi\ignorespaces}
\newcommand{\pushleft}[1]{\ifmeasuring@#1\else\omit$\displaystyle#1$\hfill\fi\ignorespaces}
\makeatother

%New definition
\def\doubleunderline#1{\underline{\underline{#1}}}

\section{Repetition}
    \subsection{Einheitskreis}    
        \begin{scriptsize}
            \begin{center}
            \begin{tabular}{|c|c|c|c|c|c|c|}
                \hline
                $\alpha$ & \thead{$\scriptstyle 0^{\degree}$ \\ $\scriptstyle [0]$} & \thead{$\scriptstyle 30^{\degree}$ \\ $\scriptstyle \left[\pi / 6\right]$} & \thead{$ \scriptstyle 45^{\degree}$ \\ $\scriptstyle [\pi / 4]$} & \thead{ $\scriptstyle 60^{\degree}$ \\ $\scriptstyle [\pi / 3]$} & \thead{$\scriptstyle 90^{\degree}$ \\ $\scriptstyle [\pi / 2]$} & \thead{$\scriptstyle 180^{\degree}$ \\ $\scriptstyle [\pi]$}\\
                \hline
                \rule[-2.5ex]{0pt}{7ex}$sin(\alpha)$ & $0$ & $\frac{1}{2}$ & $\frac{\sqrt{2}}{2}$ & $\frac{\sqrt{3}}{2}$ & $1$ & $0$\\
                \hline
                \rule[-2.5ex]{0pt}{7ex} $cos(\alpha)$ & $1$ & $\frac{\sqrt{3}}{2}$ & $\frac{\sqrt{2}}{2}$ & $\frac{1}{2}$ & $0$ & $-1$\\
                \hline
                \rule[-2.5ex]{0pt}{7ex} $tan(\alpha)$ & $0$ & $\frac{\sqrt{3}}{2}$ & $1$ & $\sqrt{3}$ & - & $0$\\
                \hline
            \end{tabular}
            \begin{empheq}[box=\fbox]{align*}
                sin^2(\alpha) + cos^2(\alpha) = 1
            \end{empheq}
            \begin{empheq}[box=\fbox]{align*}
                cos(\alpha + \frac{\pi}{2}) \: \widehat{=} \: -sin(\alpha) \quad &\mid \quad cos(\alpha -\frac{\pi}{2}) \: \widehat{=} \: sin(\alpha)
                \\ cos (\frac{\pi}{2} - \alpha) \: \widehat{=} \: sin(\alpha) \quad &\mid \quad sin(\alpha-\frac{\pi}{2}) \: \widehat{=} \: -cos(\alpha)
                \\ sin(\frac{\pi}{2} -\alpha) \: &\widehat{=} \: cos(\alpha)
            \end{empheq}
        \end{center}
        \end{scriptsize}    

    \subsection{Kreuzprodukt (Vektorprodukt)}    
        \begin{scriptsize}
            \begin{empheq}[box=\fbox]{align*}
                &a \times b = \begin{pmatrix} a_1 \\ a_2 \\ a_3 \end{pmatrix} \times \begin{pmatrix} b_1 \\ b_2 \\ b_3 \end{pmatrix} = \begin{pmatrix} a_2b_3 - a_3b_2 \\ a_3b_1 - a_1b_3 \\ a_1b_2 - a_2b_1  \end{pmatrix} \quad a,b \in \mathbb{R}^3
                \\ &  a \times b = -b \times a
                \\ & \Vert a \times b \Vert = \Vert a \Vert \cdot \Vert b \Vert \cdot sin(\measuredangle[a,b]) \: \widehat{=} \: \text{Fläche d. Parallelogramms}
            \end{empheq}
        \end{scriptsize}

    \subsection{Redutkion Linienverteilte Kraft}    
        \begin{scriptsize}
            \begin{center}
                \includegraphics[width = 0.7\linewidth]{MECH_LinienverteilteKraft}
            \end{center}    
        \end{scriptsize}

    \section{Koordinatentransformation}
        \begin{scriptsize}
            \begin{center}
                \begin{tabular}{|c|c|c|c|}
                    \hline
                    \null & \cellcolor{Lavender}Kartesisch & Zylindrisch & Sphärisch\\
                    \hline
                    \rule[-2ex]{0pt}{5ex} $x$ & \cellcolor{Lavender}$x$ & $\rho \cdot \text{cos}(\varphi)$ & $r \cdot \text{sin}(\theta)\text{cos}(\psi)$\\
                    \hline
                    \rule[-2ex]{0pt}{5ex} $y$ & \cellcolor{Lavender}$y$ & $\rho \cdot \text{sin}(\varphi)$ & $r \cdot \text{sin}(\theta)\text{sin}(\psi)$\\
                    \hline
                    \rule[-2ex]{0pt}{5ex} $z$ & \cellcolor{Lavender}$z$ & $z$ & $r \cdot \text{cos}(\theta)$\\
                    \hhline{|=|=|=|=|}
                    \rule[-2.5ex]{0pt}{7ex} $\rho$ & \cellcolor{Lavender}$\sqrt{x^2 + y^2}$ & $\rho$ & $r \cdot \text{sin}(\theta)$\\
                    \hline
                    \rule[-2.5ex]{0pt}{7ex} $\varphi$ & \cellcolor{Lavender}$\text{arctan}(\frac{y}{x})$ & $\varphi$ & $\psi$\\
                    \hline
                    \rule[-2.5ex]{0pt}{7ex} $z$ & \cellcolor{Lavender}$z$ & $z$ & $r\cdot \text{cos}(\theta)$\\
                    \hhline{|=|=|=|=|}
                    \rule[-2.5ex]{0pt}{7ex} $r$ & \cellcolor{Lavender}$\sqrt{x^2+y^2+z^2}$ & $\sqrt{\rho^2+z^2}$ & $r$\\
                    \hline
                    \rule[-2.5ex]{0pt}{8.5ex} $\theta$ & \cellcolor{Lavender}$\text{arctan} \left(\frac{\sqrt{x^2+y^2}}{z}\right)$ & $\text{arctan} (\frac{\rho}{z})$ & $\theta$\\
                    \hline
                    \rule[-2.5ex]{0pt}{7ex} $\psi$ & \cellcolor{Lavender}$\text{arctan} (\frac{y}{x})$ & $\varphi$ & $\psi$\\
                    \hline
                \end{tabular}
            \end{center}
        \end{scriptsize}    
\cbreak

    \subsection{Allgemein}
        \begin{scriptsize}
            \begin{empheq}{align*}
                &\text{Ortsvektor: } \qquad \underline{r}(t)
                \\ &\text{Geschwindigkeit: } \qquad \underline{v}(t) = \dot{\underline{r}}(t) = \dot{s} \cdot \underline{\tau}
                \\ &\text{Beschleunigung: } \qquad \underline{a} (t) = \dot{\underline{v}}(t) = \ddot{\underline{r}}(t)
                \\ &\text{Schnelligkeit: } \qquad \dot{s} = \vert \underline{v}(t) \vert = \vert \dot{\underline{r}} \vert
                \\ &\text{Tangentialer Einheitsvektor: } \qquad \underline{\tau} = \frac{\underline{v}}{\vert \underline{v} \vert}
            \end{empheq}
        \end{scriptsize}
        
        
    \subsection{Kartesisch}
        \begin{scriptsize}
            \begin{empheq}{align*}
                &\text{Ortsvektor: } \qquad \underline{r}(t) = x(t)\underline{e}_x + y(t)\underline{e}_y + z(t)\underline{e}_z
                \\ &\text{Geschwindigkeit: } \qquad \underline{v}(t) = \dot{\underline{r}}(t) = \dot{x}\underline{e}_x + \dot{y}\underline{e}_y + \dot{z}\underline{e}_z
                \\ & \text{Schnelligkeit: } \qquad \dot{s} = |\underline{v}(t)| = x(t)\underline{e}_x + y(t)\underline{e}_y + z(t)\underline{e}_z
            \end{empheq}
        \end{scriptsize}
        
    \subsection{Zylindrisch}
        \begin{scriptsize}
            \begin{empheq}{align*}
                & \text{Ortsvektor: } \qquad \underline{r}(t) = \rho(t)\underline{e}_{\rho} (\varphi(t)) + z(t) \underline{e}_z
                \\ &\text{Geschwindigkeit: } \qquad \underline{v}(t) = \dot{\underline{r}}(t) = \dot{\rho}(t)\underline{e}_{\rho} + \rho\dot{\varphi}(t)\underline{e}_{\varphi} + \dot{z}(t)\underline{e}_{z}
                \\ & \text{Schnelligkeit: } \qquad \dot{s} = |\underline{v}(t)| = \sqrt{\dot{\rho}^2 + (\rho \dot{\varphi})^2 + \dot{z}^2}
                \\ & \text{Einheitsvektor in Abhängigkeit: } \qquad \underline{e}_{\rho} = cos(\varphi)\underline{e}_x + sin(\varphi)\underline{e}_y
                \\ & \underline{e}_{\varphi} = - sin(\varphi)\underline{e}_x + cos(\varphi)\underline{e}_y \quad \mid \quad \underline{e}_z \to \text{konstant}
            \end{empheq}
        \end{scriptsize} 
        
    \subsection{Sphärisch}
        \begin{scriptsize}
            \begin{empheq}{align*}
               & \text{Ortsvektor: } \qquad \underline{r}(t) = r(t)\underline{e}_r(\theta (t), \: \psi(t))
               \\ & \text{Geschwindigkeit: } \qquad \underline{v}(t) = \dot{\underline{r}} (t)= \dot{r}(t)\underline{e}_r + r\dot{\theta}(t)\underline{e}_{\theta} + r \cdot sin(\theta)\dot{\psi}(t)\underline{e}_{\psi}
               \\ & \text{Schnelligkeit: } \qquad \dot{s} = |\underline{v}(t)| = \sqrt{\dot{r}^2 + (r\dot{\theta})^2 + (r \cdot sin(\theta)\dot{\psi})^2}  
               \\ & \text{Einheitsvektor in Abhängigkeit: } 
               \\ & \underline{e}_r = sin(\theta)cos(\psi) \underline{e}_x + sin(\theta)sin(\psi) \underline{e}_y + cos(\theta) \underline{e}_z
               \\ & \underline{e}_{\theta} = cos(\theta)cos(\psi) \underline{e}_x + cos(\theta)sin(\psi) \underline{e}_y - cos(\theta) \underline{e}_z
               \\ & \underline{e}_{\psi} = -sin(\psi)\underline{e}_x + cos(\psi) \underline{e}_y 
            \end{empheq}
        \end{scriptsize}    


    \section{Lagerbindungen und Lagerkräfte}    
    \begin{center}
        \includegraphics[width = 0.6\linewidth]{MECH_Lagerarten}
    \end{center}

    \section{Beanspruchung in geraden Balken}
        \begin{minipage}{0.3\linewidth}
            \begin{scriptsize}
                \includegraphics[width = 0.9\linewidth]{MECH_Balken}
            \end{scriptsize}
        \end{minipage}
        \begin{minipage}{0.68\linewidth}
            \begin{scriptsize}
                \begin{tabular}{|c|c|c|}
                    \hline
                    Symbol & Name & Beanspruchung auf:\\
                    \hhline{|=|=|=|}
                    $N$ & Normalkraft & \thead{\scriptsize Zug ($N > 0$) \\ \scriptsize Druck ($N<0$)}\\
                    \hline
                    $Q_2, Q_3$ & Querkräfte & Schub\\
                    \hline
                    $T$ & Torsionsmoment & Torsion\\
                    \hline
                    $M_2, M_3$ & Biegemomente & Biegung\\
                    \hline
                \end{tabular}
            \end{scriptsize}      
        \end{minipage}    

    \subsection{Bestimmung der Beanspruchung}
        \begin{scriptsize}
            \begin{enumerate}
                \item Lagerkräfte am Gesamtsystem bestimmen
                \item Körper schneide, Laufvariable und Schnittgrössen einführen
                \item Gleichgewichtsbedingungen für das abgegrenzte System aufstellen $\to$ Schnittgrössen berechnen $\to$ Momentenbedingung bzgl. Schnittpunkt!
                \item Je nach Aufgabenstellung $\to$ Beanspruchungsdiagramm zeichnen
            \end{enumerate}
        \end{scriptsize}    
\cbreak

    \subsection{Differentialbeziehungen}
        \begin{scriptsize}
            Gelten für \underline{gerade Stabträger}. $q$ steht für Kraftverteilung:
            \[
            \begin{dcases}
                Q'_y = \frac{d}{dx} Q_y = -q_y \\
                M'_z = \frac{d}{dx} M_z = -Q_y \\
                M''_z = \frac{d^2}{dx^2} M_z  =q_y
            \end{dcases}
            \quad \implies \quad 
            \begin{dcases}
                Q'_z = \frac{d}{dx}Q_z = -q_z \\
                M_y' = \frac{d}{dx} M_y = Q_z \\
                M_y'' = \frac{d^2}{dx^2} M_y = -q_z
            \end{dcases}
            \]
            \textcolor{Red}{\textbf{Wichtig:}} Niemals über unstetige Belastungen (Einzelkräfte und Einzelmomente) integrieren! \\Bestimmung der Integrationskonstanten aus folgenden Randbedingungen:  
            \begin{center}
                \vspace{2mm} \includegraphics[width = 0.7\linewidth]{MECH_Lagerart_dx}
            \end{center}
            \textbf{Beachte:}
            \\ $Q =$ positive Lagerkraft, wenn Laufvariable \underline{weg} vom Lager
            \\ $Q =$ negative Lagerkraft, wenn Laufvariable \underline{zum} Lager    
        \end{scriptsize}   
        
    \section{Beanspruchung im gekrümmten Balken}
        \begin{scriptsize}
            \begin{minipage}{0.5\linewidth}
                \includegraphics[width = 0.8\linewidth]{MECH_Beanspruchung_krummerBalken}
            \end{minipage}   
            \begin{minipage}{0.48\linewidth}
                Es lohnt sich, \textbf{in Polarkoordinaten} \\zu rechnen.
                \\Symbole sind dieselben, wie bei Abs. 4
            \end{minipage}
        \end{scriptsize} 
     
    \subsection{Bestimmung der Beanspruchung}
        \begin{scriptsize}
            \begin{enumerate}
                \item Lagerkräfte bestimmen
                \item Balken schneiden und neue Laufvariable $\varphi$ einführen \\(Achtung, hier Winkel, resp. Bogenmass!) 
                \item Beanspruchungskomponenten einführen
                \item Gleichgewichtsbedingungen aufstellen $\to$ bei Integration eine neue Integrationsvariable $\alpha$ einführen und von 0 bis $\varphi$ integrieren
            \end{enumerate}
        \end{scriptsize}
        
    \subsection{Differentialbeziehungen}  
        \begin{scriptsize}
            Die Vorzeichen stehen in direkter Beziehung zur Richtung der geführten Schnittgrössen. 
            \\\textbf{Diese gelten für gekrümmte Balken:}
            \begin{empheq}[box=\fbox]{align*}
                Q'_r - N + R \cdot q_r &= 0\\
                Q'_z + R \cdot q_z &= 0\\
                M'_r - T + R \cdot Q_z &= 0\\
                M'_z - R \cdot Q_r &= 0
            \end{empheq}
        \end{scriptsize}
     
    \section{Beanspruchungsdiagramme}
        \begin{scriptsize}
            \begin{center}
                \begin{tabular}{|c|c|c|}
                    \hline
                    \cellcolor{Salmon}Lastfall: & Einfluss der Querkraft & Einfluss auf Biegemoment\\
                    \hhline{|=|=|=|}
                    Einzelmoment & - & Sprung \\
                    \hline
                    Einzelkraft & Sprung & Linear / Knick\\
                    \hline
                    Gleichförm. Kraftverteilung & Linear & Quadratisch\\
                    \hline
                    Dreiecksverteilung & Quadratisch & Kubisch\\
                    \hline
                \end{tabular}
                \includegraphics[width = 0.8\linewidth]{MECH_Beanspruchungsdiagramm1}
            \end{center}    
        \end{scriptsize}    
    \cbreak    

    \subsection{Lösungsmethoden}
        \begin{scriptsize}
            \begin{itemize}
                \item \textbf{Knotengleichgewicht}
                \begin{enumerate}
                    \item Lagerkräfte bestimmen
                    \item Gleichgewichtsbedingungen an jedem Knoten aufstellen $\to$ Stabkräfte als Zugkräfte einführen (Pendelstütze)
                    \item Gleichgewichtssysteme auflösen $\to$ Stabkräfte $S_i$
                    \item $S>0 \to$ Belastung auf Zug
                    \\$S<0 \to$ Belastung auf Druck
                \end{enumerate}
                \item \textbf{Dreikräfteschnitt}
                \begin{enumerate}
                    \item Lagerkräfte bestimmen
                    \item An geeigneter Stelle max. 3 Stäbe durchschneiden und Stabkräfte $S_i$ einführen 
                    \item Momentengleichgewicht am Schnittpunkt zweier unbekannten Stabkräfte $\to$ Berechnung der dritten Stabkraft
                    \item Komponentenbedingung $\to$ Bestimmung der beiden anderen unbekannten Stabkräfte
                \end{enumerate}
                \item \textbf{Prinzip der virtuellen Leistung (PdvL)}
                \begin{enumerate}
                    \item Stab entfernen und Stabkraft $S_i$ als Zugkraft (+) einführen
                    \item Zulässige virtuelle Bewegung einführen, d.h. eine Bewegung einführen, die mit den kinematischen Bedingungen (Lager) des Fachwerks verträglich ist
                    \item Bestimmung der Geschwindigkeit in dne Knoten, in denen Kräfte wirken
                    \item Aus dem Prinzip der virtuellen Leistung folgt: $P = 0 \to $ Berechnung der unbekannten Stabkraft $S_i$ 
                    \\\textbf{Wichtig! Immer nur einen Stab entfernen!}
                \end{enumerate}
            \end{itemize}
        \end{scriptsize}

    \section{Spannungen}
        \begin{scriptsize}
            \begin{itemize}
                \item \textbf{Definitoin Spannung:}
                \\$\sigma :=$ Kraft pro Fläche $= Pa = \frac{N}{m^2}$
            \end{itemize}
        \end{scriptsize}
      
    \subsection{Spannungsvektor $\underline{s}$}
        \begin{scriptsize}
            \begin{center}
                \includegraphics[width = 0.7\linewidth]{MECH_Spannungsvektor}
                \begin{empheq}[box=\fbox]{align*}
                    \underline{s} &= \underline{s}(\underline{x}, \underline{n}) \qquad \qquad \underline{x} \text{ Ortsvektor, } \underline{n} \text{ Orientierung}
                    \\ &= \sigma_n + \underline{n} + \tau \cdot \underline{t} \qquad \qquad \textcolor{Red}{\underline{n} \text{ und } \underline{t} \text{ sind \underline{normiert}!}}
                    \\ &= \doubleunderline{T} \cdot \underline{n} \qquad \qquad \textcolor{Red}{\underline{n} \text{ ist \underline{normiert}}}
                    \\ \sigma_n &= (\doubleunderline{T} \cdot \underline{n}) \cdot \underline{n} \qquad \qquad \textcolor{Red}{\underline{n} \text{ ist normiert und } \sigma \text{ ist nur der Betrag}}
                    \\ &= \underline{s} \cdot \underline{n}
                    \\ \tau_n &= (\doubleunderline{T} \cdot \underline{n}) \cdot \underline{t} \qquad \qquad \textcolor{Red}{\underline{n} \text{ und } \underline{t} \text{ sind normiert und } \sigma \text{ nur Betrag}}
                    \\ &= \underline{s} \cdot \underline{t}
                    \\ &= \vert \underline{s} - \sigma_n \cdot \underline{n} \vert
                \end{empheq}
            \end{center}
        \end{scriptsize}    

    \subsection{Spannungstensor 2D $\doubleunderline{T}$}
        \begin{scriptsize}
            \begin{itemize}
                \item Ein Spannungstensor ist definiert durch einen Spannungsvektor $\underline{s}$ von drei senkrecht aufeinanderstehenden Flächenelement in einem Punkt
                \begin{empheq}[box=\fbox]{align*}
                    \text{zur Erinnerung: } \underline{s}(\underline{n}) = \doubleunderline{T} \cdot \underline{n}
                \end{empheq}
                \item \textbf{Koordinatentransformation eines 2D-Tensors:}
            \end{itemize}
            \begin{center}
                \includegraphics[width = 0.8\linewidth]{MECH_Spannungstensor}
            \end{center}
            \begin{empheq}[box=\fbox]{align*}
                \sigma_{\xi} &= cos^2(\alpha) \cdot \sigma_x + sin^2(\alpha) \cdot \sigma_y + 2 \cdot sin(\alpha)cos(\alpha) \cdot \tau_{xy}
                \\ \sigma_{\eta} &= sin^2(\alpha) \cdot \sigma_x + cos^2(\alpha) \cdot \sigma_y - 2\cdot sin(\alpha)cos(\alpha) \cdot \tau_{xy}
                \\ \tau_{\xi \eta} &= (\sigma_y - \sigma_x) sin(\alpha)cos(\alpha) + \tau_{xy} \left(cos^2(\alpha)-sin^2(\alpha)\right)
            \end{empheq}
        \end{scriptsize}
        
    \subsection{Normal- und Schubspannungen 2D}
        \begin{scriptsize}
            \begin{empheq}[box=\fbox]{align*}
                \sigma_n(\alpha) &= cos^2(\alpha) \cdot \sigma_x + sin^2(\alpha) \cdot \sigma_y + 2 \cdot sin(\alpha)cos(\alpha) \cdot \tau_{xy}
                \\ \tau_n(\alpha) &= (\sigma_y - \sigma_x) \cdot sin(\alpha)cos(\alpha) + \tau_{xy} \left(cos^2(\alpha) - sin^2(\alpha) \right)
            \end{empheq}    
        \end{scriptsize}    

    \subsection{Mohrscher Spannungskreis (hier 2D)}
        \begin{scriptsize}
            \begin{center}
                \includegraphics[width = 0.7\linewidth]{MECH_MSpannungskreis}
            \end{center}
            \begin{itemize}
                \item \textbf{Positive Normalspannungen:} (Zug $\Rightarrow \sigma > 0$) wirken in Richtung der \\Flächennormalen $\underline{n}$
                \item \textbf{Positive Schubspannungen:} (Druck $\Rightarrow \tau > 0$) wirken in Richtung von $\underline{t}$
            \end{itemize}
        \end{scriptsize}    

    \subsection{Hauptwerte und Hauptrichtungen 2D}
        \begin{scriptsize}
            \begin{itemize}
                \item \textbf{Hauptwerte $\zeta$ 2D:}
                \begin{empheq}[box=\fbox]{align*}
                    \zeta_{1,2} &= \frac{1}{2} (\sigma_x + \sigma_y) \pm \sqrt{\frac{1}{4} (\sigma_x - \sigma_y)^2 + (\tau_{xy})^2}
                    \\ &= \text{Mittelpunkt $\pm$ Radius}
                \end{empheq}
                \item \textbf{Hauptrichtungen 2D:}
                \begin{empheq}[box=\fbox]{align*}
                    &\textbf{Winkel der 1. HR: } \quad \alpha_1 = \frac{1}{2}\cdot arctan\left(\frac{2\tau_{xy}}{\sigma_x-\sigma_y}\right)
                    \\ &\textbf{1. HR im Bereich: } \quad 0 \leq \alpha_1 \leq 90^\circ
                    \\ &\to \text{wenn nicht im Bereich, mit } 90^\circ \text{ addieren od. subtrahieren} 
                    \\ &\textbf{2. HR: } \quad \alpha_2 = \alpha_1 + 90^\circ
                    \\~\\ &\to \alpha_1 \text{ in } \sigma_{\xi}(\alpha) \text{ und } \sigma_{\eta}(\alpha) \text{ einsetzen für Hauptwerte $\zeta_{1,2}$}
                \end{empheq}
                \item \textbf{Maximale Schubspannung 2D:}
                \\$\Rightarrow$ gilt bei \underline{ebenem Spannungszustand} oder wenn \underline{z} = Hauptrichtung:
                \begin{empheq}[box=\fbox]{align*}
                    \tau_{\text{max}} &= \sqrt{\frac{1}{4}(\sigma_x - \sigma_y)^2 + (\tau_{xy})^2} \quad \mid \quad \alpha_{\tau, \text{max}} = \alpha_1 \pm 90
                \end{empheq}
            \end{itemize}
        \end{scriptsize}    
     
    \subsection{Hauptwerte und Hauptrichtungen 3D}
        \begin{scriptsize}
            \begin{itemize}
                \item \textbf{Hauptwerte $\lambda_i$ 3D}:
                \\ $\lambda_i$ = Eigenwerte, $n_i$ = Eigenvektoren $\Rightarrow$ Löse $det (\doubleunderline{T}-\lambda \cdot \mathbb{I}) = 0$
            \end{itemize}
            \begin{empheq}[box=\fbox]{align*}
                &det(\doubleunderline{T} - \lambda \cdot \mathbb{I})=0 \quad \Rightarrow \quad  \lambda^3- A_1\lambda^2 - A_2\lambda-A_3 = 0
                \\ & A_1 = \sigma_x + \sigma_y + \sigma_z
                \\ & A_2 = -\sigma_x\sigma_y - \sigma_y\sigma_z-\sigma_x\sigma_z + (\tau_{xy})^2 + (\tau_{xz})^2 + (\tau_{yz})^2
                \\ & A_3 = det(\doubleunderline{T})
            \end{empheq}
            \begin{itemize}
                \item \textbf{Hauptrichtungen $n_i$ 3D:}
                    \begin{empheq}[box=\fbox]{align*}
                        \text{Löse: } \quad (\doubleunderline{T} - \lambda_i\cdot \mathbb{I}) \cdot n_i = 0
                    \end{empheq}
                \item \textbf{Maximale Schubspannung 3D:}
                    \begin{empheq}[box=\fbox]{align*}
                        \tau_{\text{max}} = \frac{1}{2}[\text{max}(\lambda_i) - \text{min}(\lambda_i)]
                    \end{empheq}
                \cbreak
                \item \textbf{weitere Beziehungen:}
            \end{itemize}
            \begin{tabular}{|c|c|c|c|c|}
                \hline
                \thead{\scriptsize Haupt- \\\scriptsize spannungs- \\ \scriptsize richtung} & $\tau$ & $-\tau$ & $\underline{n}_i$ & $\underline{t}$ \\
                \hline
                x & $\sigma_y$ & $\sigma_z$ & \thead{$\scriptstyle cos(\alpha_i)\underline{e}_y$ \\ $\scriptstyle+ sin(\alpha_i)\underline{e}_z$} & $-sin(\alpha)\underline{e}_y + cos(\alpha)\underline{e}_z$\\
                \hline
                y & $\sigma_x$ & $\sigma_z$ & \thead{$\scriptstyle cos(\alpha_i)\underline{e}_x$ \\ $\scriptstyle+ sin(\alpha_i)\underline{e}_z$} & $-sin(\alpha)\underline{e}_x + cos(\alpha)\underline{e}_z$\\
                \hline
                z & $\sigma_x$ & $\sigma_z$ & \thead{$\scriptstyle cos(\alpha_i)\underline{e}_x$ \\ $\scriptstyle+ sin(\alpha_i)\underline{e}_y$} & $-sin(\alpha)\underline{e}_x + cos(\alpha)\underline{e}_y$\\
                \hline
            \end{tabular}
            \begin{empheq}[box=\fbox]{align*}
                \sigma_{\varphi} &\simeq -\frac{R}{t} \cdot P_i
                \\ \text{mit } P_i = \text{ Innendruck; } &R = \text{ Radius; } t = \text{ Wandstärke} 
            \end{empheq}
            \begin{empheq}[box=\fbox]{align*}
                &\textbf{Determinante berechnen: } 
                \\ &\begin{pmatrix}
                    a & b & c \\
                    d & e & f \\
                    g & h & i \\ 
                \end{pmatrix} = aei + bfg + cdh - gec - hfa - idb
            \end{empheq}
        \end{scriptsize}    

    \subsection{Gleichgewichtsbedingungen}
        \begin{scriptsize}
            Wenn Komponenten vom Spannungstensor $\doubleunderline{T}$ als Funktion von $x,y,z$ gegeben sind, gelten \\die untenstehenden Beziehungen. \textbf{Ziel:} Normalspannungsverteilung zu finden. \\$f$ steht für die Raumkraftdichte, z.B. Gewichtskraft.
            \begin{empheq}[box=\fbox]{align*}
                \partial_x (\sigma_x) + \partial_y(\tau_{xy}) + \partial_z(\tau_{xz}) + f_x &= 0
                \\\partial_x(\tau_{xy}) + \partial_y(\sigma_y) + \partial_z(\tau_{yz}) + f_y &= 0
                \\ \partial_x(\tau_{xz}) + \partial_y(\tau_{yz}) + \partial_z(\sigma_z) + f_z &= 0
            \end{empheq}
        \end{scriptsize}   
        
    \section{Verzerrungen}   
        \begin{scriptsize}
            \begin{center}
                \includegraphics[width = 0.6\linewidth]{MECH_Verzerrungen}
            \end{center}
        \end{scriptsize}

    \subsection{Verzerrungstensor}
        \begin{scriptsize}
            \begin{empheq}[box=\fbox]{align*}
                \left[ \doubleunderline{E}\right]_{xy} &= \begin{pmatrix}
                    \varepsilon_x & \varepsilon_{xy} & \varepsilon_{xz}\\
                    * & \varepsilon_y & \varepsilon_{yz}\\
                    * & * & \varepsilon_z\\
                \end{pmatrix}
                \\ &= \begin{pmatrix}
                    u_{x,x} & \frac{1}{2}(u_{x,y} + u_{y,x}) & \frac{1}{2}(u_{x,z} + u_{z,x})\\
                    * & u_{y,y} & \frac{1}{2} (u_{z,y} + u_{y,z})\\
                    * & * & u_{z,z}\\
                \end{pmatrix}
            \end{empheq}
        \end{scriptsize}    

    \subsection{Schubwinkel $\gamma$}   
        \begin{scriptsize}            
            \begin{minipage}{0.58\linewidth}
                \begin{empheq}{align*}
                    \gamma_{xy} &= 2 \cdot \varepsilon_{xy}\\
                    \gamma_{xz} &= 2 \cdot \varepsilon_{xz}\\
                    \gamma_{yz} &= 2 \cdot \varepsilon_{yz}\\
                \end{empheq}
            \end{minipage}
            \begin{minipage}{0.4\linewidth}
                \begin{empheq}{align*}
                    \underline{v}(\underline{n}) &= \doubleunderline{E} \cdot \underline{n}\\
                    \varepsilon_n &= \underline{v} \cdot \underline{n}\\
                    \varepsilon_{nt} &= \vert \underline{v} - \varepsilon_n \cdot \underline{n} \vert
                 \end{empheq}
            \end{minipage}
            \[
                \varepsilon_{xy} =
            \begin{dcases}
                > 0, \; \gamma \downarrow\\
                < 0, \; \gamma \uparrow\\
            \end{dcases}
            \]
        \end{scriptsize}

    \section{Stoffgesetze}   
        \begin{scriptsize}
            \begin{empheq}[box=\fbox]{align*}
                \sigma = \frac{F}{A} \quad \mid \quad \sigma &= E \cdot \varepsilon \quad \mid \quad G = \frac{E}{2(1+v)}
                \\\Delta l = \frac{1}{E} \int\limits_{0}^{l} &\frac{N(x)}{A(x)} \ dx = \frac{F}{A\cdot E} \cdot l
            \end{empheq}
            $E$ = Elastizitätsmodul $[\frac{N}{m^2}] \to$ Beziehung zwischen $\sigma_x$ und $\varepsilon_x$;\\
            $v$ = Querdehnungszahl $[\varnothing] \to$ Poissonzahl, Materialkonstante, $0 \leq v \leq 0.5$;\\
            $G$ = Schubmodul $[\frac{N}{m^2}] \to$ Beziehung zwischen $\tau_{xy} \text{ und } \varepsilon_{xy}$\\
            $K$ = Kompressionsmodul $[\frac{N}{m^2}] \to$ Beziehung zwischen Volumendehnung und Druck 
        \end{scriptsize}
    \cbreak    

    \subsection{Temperaturabhängigkeit}
        \begin{scriptsize}
            \begin{itemize}
                \item \textbf{3D-Fall:}
            \end{itemize}
            \begin{empheq}[box=\fbox]{align*}
                \varepsilon_x &= \frac{1}{E} \left[ \sigma_x - v(\sigma_y + \sigma_z)\right] + \alpha \cdot \Delta T\\
                \varepsilon_y &= \frac{1}{E} \left[ \sigma_y - v(\sigma_x + \sigma_z)\right] + \alpha \cdot \Delta T\\
                \varepsilon_z &= \frac{1}{E} \left[ \sigma_z - v(\sigma_x + \sigma_y)\right] + \alpha \cdot \Delta T
            \end{empheq}
            \begin{empheq}[box=\fbox]{align*}
                \varepsilon_{xy} &= \frac{1}{2} \gamma_{xy} = \frac{\tau_{xy}}{2G} \quad \mid \quad \varepsilon_{yz} = \frac{1}{2} \gamma_{yz} = \frac{\tau_{yz}}{2G}\\ 
                \varepsilon_{xz} &= \frac{1}{2} \gamma_{xz} = \frac{\tau_{xz}}{2G} \quad \mid \quad \varepsilon_v = spur(\doubleunderline{E}) = \varepsilon_x + \varepsilon_y + \varepsilon_z\\
                \sigma_x &= \frac{E}{1+v} \left[ \varepsilon_x + \frac{v}{1-2v} (\varepsilon_x + \varepsilon_y + \varepsilon_z)\right] - \frac{E}{1-2v} \cdot \alpha \Delta T\\
                \sigma_y &= \frac{E}{1+v} \left[ \varepsilon_y + \frac{v}{1-2v} (\varepsilon_x + \varepsilon_y + \varepsilon_z)\right] - \frac{E}{1-2v} \cdot \alpha \Delta T\\
                \sigma_z &= \frac{E}{1+v} \left[ \varepsilon_z + \frac{v}{1-2v} (\varepsilon_x + \varepsilon_y + \varepsilon_z)\right] - \frac{E}{1-2v} \cdot \alpha \Delta T
            \end{empheq}
            \begin{itemize}
                \item \textbf{2D-Fall:}
            \end{itemize}    
            \begin{empheq}[box=\fbox]{align*}
                \varepsilon_x = \frac{1}{E}(\sigma_x - v \cdot \sigma_y) \quad &\mid \quad \varepsilon_y = \frac{1}{E}(\sigma_y - v \cdot \sigma_x)\\
                \varepsilon_{xy} &= \frac{1+v}{E} \cdot \tau_{xy}\\
                \sigma_x = \frac{E}{1-v^2} (\varepsilon_x + v\varepsilon_y) \quad &\mid \quad \sigma_y = \frac{E}{1-v^2} (\varepsilon_y + v\varepsilon_x)\\
                \tau_{xy} &= \frac{E\cdot \varepsilon_{xy}}{1+v}
            \end{empheq}
        \end{scriptsize}   
        
    \section{Balkenbiegung}
    \subsection{Schwerpunkt}
        \begin{scriptsize}
            \begin{itemize}
                \item \textbf{Körper mit einfachen (zusammengesetzten) Geometrien:}
                \begin{enumerate}
                    \item Körper in einfache Geometrien (Teilflächen od. Volumen) aufteilen
                    \item Koordinatensystem und Ursprung definieren
                    \item Teilflächen $A_i$ berechnen
                    \item Schwerpunkt einer jeden Teilfläche / Teilvolumens bestimmen
                    \item Folgende Formeln anwenden:
                \end{enumerate}
            \end{itemize}
            $$ x_s = \frac{\sum (x_i \cdot A_i)}{\sum A_i} \quad \mid \quad y_s = \frac{\sum (y_i \cdot A_i)}{\sum A_i} \quad \mid \quad z_s = \frac{\sum (z_i \cdot A_i)}{\sum A_i}$$
            \begin{itemize}
                \item \textbf{Allgemeine Formel:}
                \\ beliebiges Koordinatensystem ($e_{\eta}, e_{\xi}$):
                $$ \eta_s = \frac{1}{A} \iint_A \eta \ dA \quad \mid \quad \xi_s = \frac{1}{A} \iint_A \xi \ dA$$
                Zusammengesetzter Schwerpunkt:
                $$\eta_{\text{ges}} = \frac{1}{A_{\text{ges}}} \cdot (A_1 \cdot \Delta \eta_1 + ... + A_i \cdot \Delta \eta_i)$$
                $\Delta \eta_i$ = Abstand von $\eta = 0$ zum Schwerpunkt $\eta_i$ des Körpers \\(Abstand zur $\xi$-Achse)
            \end{itemize}
        \end{scriptsize}


    \subsection{Flächenträgheitsmomente}
        \begin{scriptsize}
            \begin{minipage}{0.4\linewidth}
                \begin{empheq}[box=\fbox]{align*}
                    I_z &= \iint_A y^2 dA\\
                    I_y &= \iint_A z^2 dA\\
                    C_{yz} &= -\iint_A yz dA\\
                    I_{\xi} &= \iint_A \eta^2 dA\\
                    I_{\eta} &= \iint_A \xi^2 dA
                \end{empheq}
            \end{minipage}
            \begin{minipage}{0.58\linewidth}
                \includegraphics[width = 0.9\linewidth]{MECH_Flaechentraegheitsmoment}
            \end{minipage}    
        \end{scriptsize} 
        \cbreak
     
    \subsubsection{Verschiebungssatz}
        \begin{scriptsize}
            \begin{minipage}{0.58\linewidth}
                \begin{empheq}[box=\fbox]{align*}
                    &\text{$y$ und $z$ im Schwerpunkt}\\
                    I_{\eta} &= I_y + (\Delta \xi)^2 \cdot A\\
                    I_{\xi} &= I_z + (\Delta \eta)^2 \cdot A\\
                    C_{\eta \xi} &= C_{yz} - \Delta \xi \cdot \Delta \eta \cdot A
                \end{empheq}
            \end{minipage}
            \begin{minipage}{0.4\linewidth}
                \includegraphics[width = 0.5\linewidth]{MECH_Verschiebungssatz}
            \end{minipage}
        \end{scriptsize}    

    \subsubsection{Trägheitsmoment im Schwerpunkt}    
        \begin{scriptsize}
            \begin{empheq}[box=\fbox]{align*}
                I_{ys} &= [I_{1y} + (\Delta z)^2 A_1] +...+ [I_{ny} + (\Delta z_n)^2 A_n]\\
                I_{zs} &= [I_{1z} + (\Delta y)^2A_1] +...+ [I_{nz} + (\Delta y_n)^2A_n]\\
                C_{yzs} &= [C_1 - \Delta \xi \cdot \Delta \eta \cdot A_1] +...+ [C_n - \Delta \xi \cdot \Delta \eta \cdot A_n]
            \end{empheq}
        \end{scriptsize}

    \subsubsection{Weitere Beziehungen}
        \begin{scriptsize}
            \begin{empheq}[box=\fbox]{align*}
                I_{\eta} &= sin^2(\alpha) I_z + cos^2(\alpha) I_y + 2sin(\alpha)cos(\alpha) C_{yz}\\
                I_{\xi} &= cos^2(\alpha) I_z + sin^2(\alpha) I_y - 2 sin(\alpha)cos(\alpha) C_{yz}\\
                C_{\eta \xi} &= cos(\alpha)sin(\alpha)(I_z-I_y) + (cos^2(\alpha) - sin^2(\alpha))\cdot C_{yz}
            \end{empheq}
        \end{scriptsize}    

    \subsection{Spezielle Flächenträgheitsmomente}
        \begin{scriptsize}
            \begin{center}
                \includegraphics[width = 0.9\linewidth]{MECH_spez_Flaechentraegheitsmomente.png}
                \begin{tabular}{|c|c|c|}
                    \hline
                    \null & \cellcolor{SkyBlue} $I_y$ & \cellcolor{Tan}$I_z$ \\
                    \hhline{|=|=|=|}
                    \rule[-2.5ex]{0pt}{7ex} 1 & \cellcolor{SkyBlue}$\frac{b\cdot h^3}{12}$ & \cellcolor{Tan}$\frac{h\cdot b^3}{12}$\\
                    \hline
                    \rule[-2.5ex]{0pt}{7ex} 2 & \cellcolor{SkyBlue}$\frac{a\cdot h^3}{36}$ & \cellcolor{Tan}$\frac{h\cdot a^3}{36}$\\
                    \hline
                    \rule[-2.5ex]{0pt}{7ex} 3 & \cellcolor{SkyBlue}$\frac{\pi}{4} \cdot (R^4-r^4)$ & \cellcolor{Tan} = $I_y$ \\
                    \hline
                    \rule[-2.5ex]{0pt}{7ex} 4 & \cellcolor{SkyBlue}$\frac{\pi}{4} \cdot (AB^3-ab^3)$ & \cellcolor{Tan}$\frac{\pi}{4} (BA^3-ba^3)$\\
                    \hline
                    \rule[-2.5ex]{0pt}{7ex} 5 & \cellcolor{SkyBlue}$h^3\frac{(b_1+b_2)^2 + 2b_1b_2}{36\cdot (b_1 + b_2)}$ & \cellcolor{Tan}$\frac{h}{48} (b_1+b_2) (b_1^2 + b_2^2)$\\
                    \hline
                    \rule[-2.5ex]{0pt}{7ex} 6 & \cellcolor{SkyBlue}$\frac{na^4}{96} \cdot \frac{2+ cos(\alpha)}{(1-cos(\alpha))^2} \cdot sin(\alpha)$ & \cellcolor{Tan} = $I_y$ \\
                    \hline
                    \rule[-2.5ex]{0pt}{7ex} 7 & \cellcolor{SkyBlue}$\frac{1}{12} \cdot (BH^3-bh^3)$ & \cellcolor{Tan}$\frac{1}{12} \cdot (HB^3-hb^3)$\\
                    \hline
                    \rule[-2.5ex]{0pt}{7ex} 8 & \cellcolor{SkyBlue}$\frac{1}{12} \cdot (BH^3-bh^3)$ & \cellcolor{Tan}$\frac{1}{12} [(H-h)B^3 + h(B-h)^3]$\\
                    \hline
                    \rule[-2.5ex]{0pt}{7ex} 9 & \cellcolor{SkyBlue}$\frac{1}{12}\cdot (BH^3-bh^3)$ & \cellcolor{Tan}-\\
                    \hline
                \end{tabular}
            \end{center}
        \end{scriptsize}    

    \cbreak
    \subsection{Allgemeine Biegung \hfill (y und z sind Hauptachsen)}  
        \begin{scriptsize}
            \begin{itemize}
                \item \textbf{Normalspannung in x-Richtung:}
                \begin{empheq}[box=\fbox]{align*}
                    \sigma(x,y,z) &= \frac{N(x)}{A} - \frac{M_z(x)}{I_z} \cdot y + \frac{M_y(x)}{I_y} \cdot z
                \end{empheq}
                \item \textbf{DGL für Mittellinie:}
            \end{itemize}
            \begin{empheq}[box=\fbox]{align*}
                u_{x0}'(x) = \frac{N(x)}{EA} \quad &\mid \quad v_{y0}'(x)= \frac{M_z(x)}{EI_z} 
                \\ w_{z0}'(x) &= -\frac{M_y(x)}{EI_y}
            \end{empheq}
            \begin{itemize}
                \item \textbf{Dehnung in x-Richtung:}
                \begin{empheq}[box=\fbox]{align*}
                    \varepsilon(x,y,z) = u_0'(x) - y \cdot v_0''(x) - z \cdot w_0''(x)
                \end{empheq}
                \item \textbf{DGL für Querkraft und Biegemoment:}
                \begin{empheq}[box=\fbox]{align*}
                    M_z'(x) = - Q_y(x) \quad \mid \quad M_y'(x) = Q_z(x)
                \end{empheq}
            \end{itemize}
        \end{scriptsize}

    \subsection{Schiefe Biegung}
        \begin{scriptsize}
            \begin{minipage}{0.4\linewidth}
                \includegraphics[width = 1.0\linewidth]{MECH_schiefeBiegung.png}
            \end{minipage}
            \begin{minipage}{0.58\linewidth}
                \begin{empheq}[box=\fbox]{align*}
                    \sigma_x = \frac{M_y}{I_y} \cdot z - \frac{M_z}{I_z} \cdot y
                \end{empheq}
            \end{minipage}
        \end{scriptsize}   
        
    \subsection{Y, Z $\Rightarrow$ keine Hauptachsen}    
        \begin{scriptsize}
            \begin{empheq}[box=\fbox]{align*}
                &M_2, M_3, I_3, I_2 \text{ durch Projektion finden}\\
                &u_{x0}, u_{y2}, u_{z3} \text{ in Richtung der Hauptachsen}\\
                &u_{x0}'(x) = \frac{N(x)}{EA} \quad \mid \quad u_{y2}''(x) = \frac{M_3(x)}{EI_3}\\ 
                &u_{z3}''(x) = -\frac{M_z(x)}{EI_2}\\
                &\sigma(x,y,z) = \frac{N(x)}{A}-\frac{M_3(x)}{I_3} \cdot y + \frac{M_2(x)}{I_2}\cdot z\\
                &\text{Rückprojektion von $u_0, u_2, u_3$ um Verschiebung in x,y,z-Richtung zu finden}
            \end{empheq}
        \end{scriptsize}

    \section{Schubspannung infolge Biegung}
    \subsection{Vollquerschnitt}
        \begin{scriptsize}
            \begin{minipage}{0.58\linewidth}
                \begin{empheq}[box=\fbox]{align*}
                    &\tau_{xy} (x,y) = \frac{Q_y(x)}{I_z} \cdot \frac{H_z(y)}{b(y)}\\
                    &H_z(y) = \int\limits_{y}^{y_{\text{max}}} \eta \cdot b(\eta) \ d\eta\\ 
                    &\text{$y$ im Schwerpunkt d. Körpers}
                \end{empheq}
            \end{minipage}
            \begin{minipage}{0.4\linewidth}
                \includegraphics[width = 1.0\linewidth]{MECH_Vollquerschnitt_Schubspannung.png}
            \end{minipage}
        \end{scriptsize}
      
    \subsection{Offen, dünnwandig}
        \begin{scriptsize}
            \begin{minipage}{0.58\linewidth}
                \begin{empheq}[box=\fbox]{align*}
                    &\tau_{xs}(x,s) = -\frac{Q_y(x)}{I_z} \cdot \frac{H_z(s)}{e(s)}\\
                    &H_z(s) = \int\limits_{0}^{s} y(\eta) \cdot e(\eta) \ d\eta\\
                    &\text{Polarkoordinaten} \to d\eta = r \cdot d\varphi
                \end{empheq}
            \end{minipage}
            \begin{minipage}{0.4\linewidth}
                \includegraphics[width = 0.8\linewidth]{MECH_offen_duennwandig.png}
            \end{minipage}
        \end{scriptsize}   
        
    \subsection{Schubmittelpunkt}
        \begin{scriptsize}
            \begin{minipage}{0.58\linewidth}
                \begin{empheq}[box=\]{}
                    
                \end{empheq}
            \end{minipage}
        \end{scriptsize}    







\end{document}
