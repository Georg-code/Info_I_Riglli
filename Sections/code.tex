
\definecolor{myBlue}{HTML}{3a0077}
\definecolor{mysubBlue}{HTML}{5710a3}
\definecolor{mysubsubBlue}{HTML}{883ada}


\section{Snippets}





\subsection{Overflow Check}

\begin{lstlisting}[style = codeexpert]
bool additionWillOverflow(int a, int b) {
    if (b > 0 && a > INT_MAX - b) return true;
    if (b < 0 && a < INT_MIN - b) return true;
    return false;
}
\end{lstlisting}






\subsubsection{Decrypt (shift)}


\begin{lstlisting}[style = codeexpert]
char decrypt_char(const char encrypted_char, int shift) {
  char decrypted_char = encrypted_char;
  if (encrypted_char >= 'a' && encrypted_char <= 'z') {
    decrypted_char = (encrypted_char - 'a' + shift) % 26 + 'a';
  }

  return decrypted_char;
}
\end{lstlisting}

\subsubsection{Bubble sort}
\begin{lstlisting}[style = codeexpert]
    for (size_t i = 0; i < vec.size() - 1; ++i) {
        for (size_t j = 0; j < vec.size() - i - 1; ++j) {
            if (vec[j] > vec[j + 1]) {
                std::swap(vec[j], vec[j + 1]); // Swap elements
            }
        }
    }
\end{lstlisting}

\subsubsection{Function using istream}
\begin{lstlisting}[style = codeexpert]
//formats input into vector
  std::vector<char> alphabet_fromistream(std::istream& in) {
  // input format: 4 a b c d 
  unsigned int size;
  in >> size;
  std::vector<char> a(size);
  for (unsigned int i = 0; i < size; ++i)
    in >> a[i];
  return a; // [a b,c,d]
}
\end{lstlisting}

\subsubsection{Stream funktion operator overloading }
\begin{lstlisting}[style = codeexpert]
std::ostream& operator<<(std::ostream& out, const DualNumber& dn) {
  return out << "(" << dn.a 
             << ((dn.b < 0.0) ? " - " : " + ") 
             << "e*" << std::abs(dn.b) << ")";
}

std::istream& operator>>(std::istream& in, DualNumber& dn) {
  return in >> std::skipws >> dn.a >> std::ws >> dn.b;
}
\end{lstlisting}

\subsubsection{Turn number (123) to (321)}
\begin{lstlisting}[style = codeexpert]
int reverseNumber(int num) {
    int reversed = 0;
    while (num > 0) {
        reversed = reversed * 10 + num % 10;
        num = num / 10;
    }
    return reversed;
}
\end{lstlisting}


\subsection{Vectors and Matrices}
\subsubsection{Read/Print Vector}
\begin{lstlisting}[style = codeexpert]
using vec = std::vector<int>;
void read_vector(vec& v) {
  int size;
  std::cin >> size;
  v = vec(size);
  for (int i = 0; i < int(v.size()); ++i) {
    std::cin >> v[i];
  }
}

void print_matrix(const mat& m) {
  std::cout << "m " << m.size() << " " << m[0].size() << "\n";
  for (int row = 0; row < int(m.size()); ++row) {
    for (int col = 0; col < int(m[row].size()); ++col) {
      std::cout << m[row][col] << " ";
    }
    std::cout << "\n";
  }
}
\end{lstlisting}

\subsubsection{Read/Print Matrix}
\begin{lstlisting}[style = codeexpert]
using vec = std::vector<int>;
using mat = std::vector<vec>;

    void read_matrix(mat& m) {
  int n_rows, n_cols;
  std::cin >> n_rows >> n_cols;

  m.resize(n_rows);
  for (int row = 0; row < int(m.size()); ++row) {
    m[row].resize(n_cols);
    for (int col = 0; col < int(m[row].size()); ++col) {
      std::cin >> m[row][col];
    }
  }
}

void print_matrix(const mat& m) {
  std::cout << "m " << m.size() << " " << m[0].size() << "\n";
  for (int row = 0; row < int(m.size()); ++row) {
    for (int col = 0; col < int(m[row].size()); ++col) {
      std::cout << m[row][col] << " ";
    }
    std::cout << "\n";
  }
}
\end{lstlisting}






\subsubsection{FindMax in Vector}
\begin{lstlisting}[style = codeexpert]
int findMax(const std::vector<int>& vec) {
    if (vec.empty()) return 0; // Handle empty vector case
    int maxVal = vec[0];
    for (int i = 1; i < vec.size(); i++) {
        if (vec[i] > maxVal) {
            maxVal = vec[i];
        }
    }
    return maxVal;
}
\end{lstlisting}


\subsubsection{GRID: Neighbors Exist? Bounds check}
\begin{lstlisting}[style = codeexpert]
bool checkAllNeighborsExist(const std::vector<std::vector<int>>& grid, int row, int col) {
    int rows = grid.size();
    int cols = grid[0].size();
    
    int directions[8][2] = {{-1, -1}, {-1, 0}, {-1, 1}, {0, -1}, {0, 1}, {1, -1}, {1, 0}, {1, 1}};
    
    for (auto& dir : directions) {
        int newRow = row + dir[0];
        int newCol = col + dir[1];
        if (newRow < 0 || newRow >= rows || newCol < 0 || newCol >= cols) {
            return false; // Neighbor is out of bounds
        }
    }
    return true;
}
\end{lstlisting}

\subsection{Trees}
\subsubsection{FindMax}
\begin{lstlisting}[style = codeexpert]
// POST: Find and return the maximum leaf node value in a (sub)tree starting 
//       at root. If root is the empty tree, return 0.
unsigned int findMax(const Node* root) {
  unsigned int max = 0;
  
  // If the tree is empty, return 0
  if (root == nullptr) {
    return 0;
  }
  
  // If the current node is a leaf, return its value
  if (root->isLeaf()) {
    return root->value;
  }
  
  // Otherwise, the current node is not a leaf, and we search for the max 
  // in its children
  for (Node* child : root->children) {
    unsigned int childMax = findMax(child);
    
    if (childMax > max) {
      max = childMax;
    }
  }
  
  return max;
}
\end{lstlisting}

\subsubsection{Deletion of subtree}
\begin{lstlisting}[style = codeexpert]
// PRE:  subroot not null
// POST: subtree with root subtree memory leak free deleted
void delete_subtree(Node* subroot) {
  if (subroot->left != nullptr) {
    delete_subtree(subroot->left);
    subroot->left = nullptr;
  }
  
  if (subroot->right != nullptr) {
    delete_subtree(subroot->right);
    subroot->right = nullptr;
  }
  
  delete subroot;
}
\end{lstlisting}

\subsubsection{Prune}
\begin{lstlisting}[style = codeexpert]
// PRE:  value != 0
// POST: Delete a leaf node with given value from the (sub)tree starting at root.
//       If a leaf node is deleted, remove it from its parent.
//       If an inner node becomes empty, remove it from its  parent and delete it.
//       Return true if and only if the (sub)tree starting at root was deleted.
bool deleteLeaf(Node* root, unsigned int value) {
  if (root == nullptr) {
    return false;
  }
  if (root->isLeaf() && root->value == value ) {
    delete root;
    return true;
  }
  for (unsigned int i = 0; i < root->children.size(); ++i) {
    if (deleteLeaf(root->children[i], value)) {
      root->removeFromChildren(root->children[i]);
      
      if (root->children.size() == 0) {
        delete root;
        
        return true;
      }
    }
  }
  return false;
}
\end{lstlisting}



\subsubsection{Tree Depth}
\begin{lstlisting}[style = codeexpert]
VARIANT 1
struct TreeNode {
  unsigned int data; //INV: between 0 and 9
  TreeNode *left;
  TreeNode *right;
};
/***
 * PRE: node is a valid pointer to an instance for kind TreeNode, and not nullptr
 * POST: returns the maximum depth of the binary tree starting from node
 * 
 * Examples:
 *      3     -> 1
 *
 *      5
 *     / \ 
 *    2   7   -> 3
 *       /
 *      4
 ***/
unsigned int get_max_depth(TreeNode *node){
  assert(node != nullptr);
  unsigned int depthRight = 0;
  unsigned int depthLeft = 0;
  if (node->left){
    depthLeft = get_max_depth(node->left);
  }
  if (node->right){
    depthRight = get_max_depth(node->right);
  }
  if (depthLeft > depthRight){
    return depthLeft + 1;
  } else {
    return depthRight + 1;
  }
}

VARIANT 2
unsigned int noOfLevels(Node* node) {
  // TO DO
  if (node == nullptr)
    return 0;
  return std::max(noOfLevels(node->left), noOfLevels(node->right)) + 1;
}


\end{lstlisting}


\subsection{Linked List}
\subsubsection{Insert Element in Linked list}
\begin{lstlisting}[style = codeexpert]
void insertAtValue(Node*& head, int value, int pos) {
    if (!head) {
        head = new Node(value);
        return;
    }
    Node* temp = head;

    // Insert at the beginning if pos is 0
    if (pos == 0) {
        Node* newNode = new Node(value);
        newNode->next = head;
        head = newNode;
        return;
    }
    // Insert at the random position
    temp = head;
    for (int i = 0; i < pos - 1; i++) {
        temp = temp->next;
    }
    
    Node* newNode = new Node(value);
    newNode->next = temp->next;
    temp->next = newNode;
}
\end{lstlisting}
\columnbreak
\subsubsection{Deconstruct Linked list}
\begin{lstlisting}[style = codeexpert]
void deconstructList(Node* head) {
    if (head == nullptr) return;  // Base case: If head is null, return
    deconstructList(head->next); // Recursive call to delete next node first
    delete head;
}
\end{lstlisting}

\subsubsection{Delete Node with Value}
\begin{lstlisting}[style = codeexpert]
void deleteByValue(Node*& head, int value) {
    if (!head) {
        return; // If the list is empty, return
    }
    // Special case: if the node to be deleted is the head
    if (head->data == value) {
        Node* temp = head;
        head = head->next;
        delete temp;
        return;
    }
    // Traverse the list to find the node with the value
    Node* prev = head;
    Node* curr = head->next;
    while (curr) {
        if (curr->data == value) {
            prev->next = curr->next;
            delete curr;
            return;
        }
        prev = curr;
        curr = curr->next;
    }
}
\end{lstlisting}
% \subsubsection{Deconstruct Linked list}
% \begin{lstlisting}[style = codeexpert]
% void deconstructList(Node* head) {
%     if (!head) return;  // Base case: If head is null, return
%     deconstructList(head->next); // Recursive call to delete next node first
%     cout << "Deleting node with value: " << head->data << endl;
%     delete head;
% }

% \end{lstlisting}

\subsection{Numbers}
\subsection{Int to string}

\begin{lstlisting}[style = codeexpert]
std::string intToString(int num) {
    std::string result = "";
    bool isNegative = (num < 0);
    num = isNegative ? -num : num;
    do {
        result = char('0' + (num % 10)) + result;
        num /= 10;
    } while (num > 0);
    return isNegative ? "-" + result : result;
}
\end{lstlisting}
\subsection{Integer rounding}
\begin{lstlisting}[style = codeexpert]
int a = 7, b = 3;
int flooring = a / b; // flooring
int ceiling = (a + b - 1) / b; // ceiling
int nearest = (a + b / 2) / b; // nearest
\end{lstlisting}

\subsubsection{Decimal to Binary}
\begin{lstlisting}[style = codeexpert]
// PRE: decimal is a non-negative integer (unsigned int)
// POST: Binary representation of decimal is printed to standard output

void binary_conv(unsigned int decimal) {
    unsigned int m = 0;  // Exponent
    unsigned int power = 1;  // Power of two

    for (; power <= decimal / 2; m++) {
        power *= 2;
    }  // Find the highest power of two

    for (unsigned int i = 0; i <= m; ++i) {
        std::cout << decimal / power;
        decimal %= power;
        power /= 2;
    }  // Output the binary representation bit by bit
}

\end{lstlisting}
\columnbreak
\subsubsection{Base Conversion}
\begin{lstlisting}[style = codeexpert]
// PRE: A number >= 0 in base 10, and a base 1 < b < 10
//
// POST: Return the number in base b, i.e., return the number whose digits 
//       are the digits of the representation of the input number n in base b
//
// Examples:
//    * 10 2 ~~> 1010
//    * 14 9 ~~> 15
unsigned int convert_base(unsigned int n, unsigned int b) {
  unsigned int basetenposition = 1;
  unsigned int num = 0;
  while(n!=0){
    num += (n%b)*(basetenposition);
    n = n /b;
    basetenposition*=10;
  }
  return num;
}

\end{lstlisting}

\subsubsection{Binary Expansion}
\begin{lstlisting}[style = codeexpert]
// Converts a fractional part of a number (0 <= x < 2) into its binary representation
void convert_to_binary(double x) {
    // Check if input is within the valid range
    if (x < 0.0 || x >= 2.0) {
        std::cout << "Input out of bounds\n";
        return;
    }
    // Determine the initial bit based on x and output it
    int b = (x >= 1) ? 1 : 0;
    std::cout << b << ".";
    // Convert the fractional part to binary representation (up to 15 bits)
    for (int i = 1; i < 16; ++i) {
        x = (x - b) * 2;  // Subtract the current bit value and multiply by 2
        b = (x >= 1) ? 1 : 0;  // Determine the next binary bit
        std::cout << b;
    }
    std::cout << "\n";
}


\end{lstlisting}

\subsubsection{Prime number}
\begin{lstlisting}[style = codeexpert]
bool isPrime(int n) {
    if (n <= 1) return false;
    for (int i = 2; i < n; i++) {
        if (n % i == 0) {
            return false;
        }
    }
    return true;
}
\end{lstlisting}

\subsubsection{Greatest Common Denominator, Least Common Multiple}
\begin{lstlisting}[style = codeexpert]
using namespace std;

int gcd(int a, int b) {
    while (b != 0) {
        int temp = b;
        b = a % b;
        a = temp;
    }
    return a;
}

int lcm(int a, int b) {
    return (a / gcd(a, b)) * b;
}
\end{lstlisting}
\columnbreak
\subsection{Recursion}
\subsubsection{Bitstrings up to n}
\begin{lstlisting}[style = codeexpert]
// Recursive function to generate all bitstrings up to length n
void generate_bitstrings(int n, std::string current, std::vector<std::string>& result) {
    // Base case: Add the current bitstring to the result
    result.push_back(current);

    // If current length is less than n, continue generating longer bitstrings
    if (current.length() < n) {
        generate_bitstrings(n, current + "0", result);
        generate_bitstrings(n, current + "1", result);
    }
}

std::vector<std::string> all_bitstrings_up_to_n(int n) {
    std::vector<std::string> result;
    // Start the recursion with an empty string
    generate_bitstrings(n, "", result);
    return result;
}
\end{lstlisting}

\subsubsection{Queens on chessboard}
(Just for basic idea of backtracking)

\begin{lstlisting}[style = codeexpert]
bool solveNQueens(std::vector<std::vector<int>>& board, int col, int n) {
    if (col >= n) return true;  // Base case: If all queens are placed, return true

    for (int row = 0; row < n; row++) {
        if (isSafe(board, row, col, n)) {
            board[row][col] = 1;  // Place queen
            if (solveNQueens(board, col + 1, n)) return true;  // Recur to place next queen
            board[row][col] = 0; // Backtrack if placing queen here didn’t work
        }
    }
    return false;  // No valid placement found in this column
}

\end{lstlisting}

\subsubsection{Rod}
\begin{lstlisting}[style = codeexpert]
// pre: p.size() >= length + 1
// post: return the best value that can be attained by
//       cutting a rod of the given length and given
//       prices per length
unsigned int bestValue(const prices& p, unsigned int length){
    if(length <=0){
    return 0;
  }
  int max = 0;
  for (unsigned k = 1; k <= length; k++) {
    int value = p[k] + bestValue(p, length-k);
    if(value > max){
      max = value;
    }
  }
  return max;
}

\end{lstlisting}
\columnbreak
\subsubsection{All permutations}
\begin{lstlisting}[style = codeexpert]
StringSet all_permutations_2(const std::string& s) {
  int n = s.size();
  // base case: empty string
  if (n == 0) {
    return StringSet("");
  }
  // init empty result
  StringSet result;
  // split string into first character and remaining characters
  std::string first_character = s.substr(0, 1);
  std::string remaining_string = s.substr(1);
  // generate all permuations of remaining string
  StringSet all_remaining_permutations = all_permutations_2(remaining_string);
  // insert first character at every location in every permutation
  for (int j = 0; j < all_remaining_permutations.size(); ++j) {
    for (int i = 0; i < n; ++i) {
      result.insert(all_remaining_permutations[j].substr(0, i) + 
                    first_character +
                    all_remaining_permutations[j].substr(i));
    }
  }
  return result;
}
\end{lstlisting}

\subsection{Smart Pointers}
\begin{lstlisting}[style = codeexpert]
class Demo {
public:
    int value;
    Demo(int val) : value(val) {
        std::cout << "Demo object created with value: " << value << std::endl;
    }
    ~Demo() {
        std::cout << "Demo object with value " << value << " destroyed" << std::endl;
    }
    void show() {
        std::cout << "Value: " << value << std::endl;
    }
};

void uniquePtrExample() {
    std::cout << "\n=== Unique Pointer Example ===\n";
    // Creating a unique pointer
    std::unique_ptr<Demo> uniqueDemo = std::make_unique<Demo>(10);
    uniqueDemo->show();
    
    // Transferring ownership using std::move
    std::unique_ptr<Demo> movedDemo = std::move(uniqueDemo);
    if (!uniqueDemo) {
        std::cout << "uniqueDemo is now null after move\n";
    }
    movedDemo->show();
}  // movedDemo goes out of scope and deletes the object

void sharedPtrExample() {
    std::cout << "\n=== Shared Pointer Example ===\n";
    std::shared_ptr<Demo> shared1 = std::make_shared<Demo>(20);
    {
        std::shared_ptr<Demo> shared2 = shared1; // Shared ownership
        std::cout << "Shared pointer count: " << shared1.use_count() << std::endl;
        shared2->show();
    } // shared2 goes out of scope, shared1 is still valid
    
    std::cout << "Shared pointer count after inner scope: " << shared1.use_count() << std::endl;
}  // shared1 goes out of scope and deletes the object


\end{lstlisting}

\subsection{Overlapping intervals}

\begin{lstlisting}[style = codeexpert]
// PRE: [a1, b1], [a2, b2] are (generalized) intervals,
//      with [a, b] := [b, a] if a > b
// POST: returns true if [a1, b1], [a2, b2] overlap
bool intervals_overlap(int a1, int b1, int a2, int b2) {
    return max(a1, b1) >= min(a2, b2) 
           && min(a1, b1) <= max(a2, b2);
}
\end{lstlisting}


\section{Every little thing is gonna be alright, just copy, be happy}

\subsubsection{number into digits}
\begin{lstlisting}[style = codeexpert]
std::vector<int> splitDigits(int num) {
    std::vector<int> digits;
    
    // Handle negative numbers
    if (num < 0) {
        num = -num;
    }

    // Extract digits from the number
    while (num > 0) {
        digits.push_back(num % 10); // Get the last digit
        num /= 10; // Remove the last digit
    }

    // Reverse the vector to get digits in correct order
    std::reverse(digits.begin(), digits.end());

    return digits;
}

\end{lstlisting}


\subsubsection{Integral approx}
\begin{lstlisting}[style = codeexpert]
double integral_approx(double lb, double ub,
                       const std::function<double(double)>& func,
                       double delta = .0001) {
    double result = 0;
    uint64_t numDeltas = static_cast<uint64_t>((ub - lb) / delta);
    for (int i = 0; i < numDeltas; i++) {
        double begin = lb + i * delta;
        double end = lb + (i + 1) * delta;
        result += delta * (func(begin) + func(end)) / 2;
    }
    return result;
}


\end{lstlisting}

\subsubsection{Reverse a linked list}
\begin{lstlisting}[style = codeexpert]
// Function to reverse a linked list recursively
void reverse(Node *previous, Node *current) {
    // Base case: If we reach the last node
    if (current->next == nullptr) {
        current->next = previous;  // Point the last node to the previous node
        previous->next = nullptr;  // Ensure the old link is broken
        start = current;           // Update the head of the list
        return;
    } else {
        // Recursive call moving forward in the list
        reverse(current, current->next);
        
        // After recursion unwinds, update the links
        current->next = previous;
        previous->next = nullptr;  // Ensure the old link is broken
    }
}

\end{lstlisting}